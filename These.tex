\documentclass[10pt,a4paper]{article}


\begin{document}
	\begin{titlepage}
		\begin{center}
		\Large Mémoire de Thèse\\
		\vspace{4em}
		\LARGE\textbf{Visualisation et Interactions avec une colonie d'abeilles virtuelle :\\}
		\vspace{1em}
		\Large\textbf{Simulation, pédagogie et complexité.\\}		
		\vspace{8em}
		\LARGE Thomas Alves\\
		\vspace{8em}
		\Large Dirigé par Thierry Duval et Vincent Rodin\\
		Encadré par Jérémy Rivière\\
		\vspace{8em}
		\Large UBO IMT-A LABSTICC\\
		\end{center}	 
	\end{titlepage}
	
\tableofcontents
	
\section{Introduction}
\begin{itemize}
	\item Colonie d'abeille en tant que système complexe. Beaucoup de recherche sur les butineuses mais moins sur l'intérieur de la ruche.	
	\item Comprendre l'auto organisation interne par la modélisation SMA vs Équation différentielles - de l'importance des contacts individuels.	
	\item Transmettre et faciliter l'apprentissage avec l'environnement immersif.
\end{itemize}
\section{Etat de l'art: Simulation Multi-Agents Interactives}
	\subsection{SMA : Rercéer et comprendre des systemes complexes existants}
		Comprendre les mécanismes, créer un modèle puis évaluer l'impact des différents paramètres sur l'évolution du comportement du système.
	\subsection{Manipuler et observer ces systèmes complexes}
		Systèmes en général non immersif, on s'arrête à la RA, et [de ce que j'ai vu], sans utilsation d'interacteurs tangibles.
\section{État de l'art: Modèles Multi-Agents et prise de décision}
	Etat de l'art de l'article PAAMS
	\subsection{Foraging for work}
	\subsection{Modèles à seuils}
	\subsection{Motivation, un pas vers les BDI ?}
		Pas encore fait mais le parallèle est peut-être intéressant ?
\section{Proposition Modèle de prise de décision PAAMS}
	Description du modèle d'interruption PAAMS
	\subsection{Modélisation des tâches : Tâches, Activités, Actions et Subsomption Hiérarchiques}
	\subsection{Interruption : Motivation}
		\subsubsection{Etat de l'art rapide sur la Motivation}
		\subsubsection{Notre utilisation de la Motivation}
	\subsection{Sélection : Modèle à seuil}
	\subsection{Utilisation en Swarm Robotics}
		L'exemple avancé dans l'article PAAMS pourrait être poussé un peu grâce au travail du stagiaire et à des nouveaux trucs que j'ai compris pendant la paramétrisation complexe du bousin.
			
\section{Simulation Multi-Agents d'une colonie d'abeilles}
	\subsection{Biologie : Connaissances des biologistes sur l'abeille}
		Synthèse des connaissances, notamment de notre visite à Avignon et ses 43 degrés à l'ombre.\\
		
		Rôles et fonctions physiologiques associées.\\
		Différents rôles de différentes phéromones.\\
		Modélisation simplifiée : \\
		Cas classique de la vie d'une abeille.\\
		Quels changements provoquent quelles réactions dans le métabolisme de l'abeille, et donc dans la répartition du travail.
	\subsection{Description du modèle adapté}
		Application du modèle théorique à notre simulation de colonie d'abeilles virtuelle.
		Simplification gigantesque de la biologie de l'abeille, ainsi que des différentes tâches au sein de la colonie.
		On veut de la répartition des tâches dans le plus simple des contextes.
	\subsection{Description de l'implémentation}
		L'implémentation du modèle de prise de décision décris dans la partie d'avant, et modèle biologique de l'abeille adulte avec le systeme a seuil.
	\subsection{Calibration}
		\subsubsection{Calibration rapide - Accélération}
			Dans sa version PAAMS avec la biologie accélérée
			Obtenir des résultats dans trop de temps de simulation. Le but était de vérifier la répartition des taches a l'aide du modèle, pas encore de chercher une validité biologique.
		\subsubsection{Calibration biologique - Données directes et indirectes}
			En ajustant les paramètres connus grâce à la biologie, comme les durées de vie ou la durée et fréquence des nourrissements par exemple.
			Paramétristaion des paramètres indirects, ou émergents, comme par exemple le fameux "UNE nourrice s'occupe de DEUX larves", qui n'est codé nul part mais qu'on doit retrouver. [Spoiler alert on le retrouve pas, et je ne sais pas pourquoi]
	\subsection{Analyse des résultats}
		Ce qui marche, ce que ça montre et ce qui ne marche pas, et ce que ça peut vouloir dire.
		
			
\section{État de L'art VISU INTER}
	Comment mieux comprendre un système complexe : Environnement immersif et interacteurs tangibles. Ruche, cadre et connaissances apicoles.
	\subsection{DataViz}
		Comment visualiser de grande quantité de données, les données micro.
	\subsection{Interaction / visu immersive pour comprendre}
		Environnement immersif
	\subsection{Interaction tangible}
		Interacteurs tangibles
\section{Proposition visu interaction}
	Notre proposition
	\subsection{Interaction Immersive avec Manettes}
		Manipulation des cadres avec les manettes
	\subsection{Interaction Immersice ET tangible}
		L'apport et les contraintes du tangible (qu'il faut encore définir) pour la manipulation des cadres
	\subsection{Visualisation : Graph3D sur l'état interne de la colonie}
		Le graph3D et les information qu'il apporte
	\subsection{Résultats/Évaluation Visualisation Interactive proposée}
\section{Discussions}
	Comment l'ensemble se comporte et avis critique sur la totalité. Notamment le modèle multi agent simplifié qui apporte une quantité gigantesque de biais.
\section*{Conclusion}
	C'était cool
\section*{Perspectives}
	Tout est possible, pousser le modèle de l'abeille de la simulation, pousser le domaine tangible avec peut être un cadre manette (un cadre avec un ou deux boutons?)
\section*{Rideau}
	clapclapclapclapclapclapclapclapclapclap

\end{document}