\chapter{Etat de l'art: Simulation Multi-Agents et Environnements Immersifs}

	Dans ce chapitre nous présentons différentes approches utilisées afin d'interagir avec des systèmes multi-agents, ainsi que différentes approches pour les en visualiser différentes propriétés. Nous étendons ensuite nos observations à l'interaction et la visualisation en ce qui concerne les environnement immersifs, où l'utilisateur est plongé au cœur de l'application, souvent grâce à un casque de réalité virtuelle.

	\section{Visualiser et Interagir avec une Simulation Multi-Agents}
		\subsection{Simuler un Système pour Améliorer notre Compréhension}
		
		Simuler un système permet de mieux en comprendre les différentes propriétés. Tel un dessinateur qui se voit obligé d'analyser la moindre forme de son modèle, la simulation (et/ou la modélisation) force l'auteur à se pencher sur le moindre de ses mécanismes. Le modèle, une fois mis en place, peut alors être confronté à la réalité, souvent bien plus complexe, plus subtile. Ensuite, cette modélisation est un excellent outil pour vulgariser les concepts qui ont du être assimilés, voire créés, pour la concevoir. La modélisation peut aussi permettre d'en manipuler les paramètres, afin d'anticiper le comportement de son homologue réel.
		
	\boitemagique{Jacques Tisseau, HDR 2001 \cite{tisseau_realite_2001}}{"\textit{Les fonctions d'un modèle sont multiples. Pour le modélisateur, le modèle permet d'imaginer, de concevoir, de prévoir et d’améliorer sa représentation du phénomène ou de l'idée qu'il cherche à modéliser. Le modèle devient un support de communication pour représenter, sensibiliser, expliquer ou enseigner les concepts concernés. A l'autre bout de la chaîne, le modèle aide l'utilisateur dans sa compréhension du phénomène (ou de l'idée) représenté(e); il peut également l'évaluer et l'expérimenter par simulation.}"}
	
	Le projet \textit{SIMBACA} s'inscrit dans l'ensemble de cette démarche, de la construction du modèle de la colonie à la vulgarisation aux apiculteurs, en passant par la production d'un outil pour les biologistes de l'abeille.
		
		
		\subsection{Visualisation de Systèmes Multi-Agents}
		
		Louloudi et. al. \cite{louloudi_new_2012} ont proposé dans leurs travaux de séparer la partie simulation multi-agents de la partie visualisation de leur programme. Cette séparation offre de pouvoir faire tourner le modèle sans les contraintes de la visualisation (souvent le respect du temps réel) lorsque la visualisation n'est pas nécessaire. Ils proposent un composant logiciel "relai" entre la simulation et la visualisation agissant comme un convertisseur universel. Or, les granularité de temps et d'espaces peuvent varier fortement d'un modèle multi-agents à un autre, ainsi que le type même de données à transmettre, ce qui, couplé à un besoin de transmettre en retour des informations depuis l'application de visualisation vers le simulateur rendent cette notion de relai \textit{universel} bien difficile à mettre en place.
		
		
		\subsection{Interactions avec des Systèmes Multi-Agents}
		Systèmes en général non immersif, on s'arrête à la RA, et [de ce que j'ai vu], sans utilisation d'interacteurs tangibles.
	Comment mieux comprendre un système complexe : Environnement immersif et interacteurs tangibles. Ruche, cadre et connaissances apicoles.
	
	\subsubsection{Le Modèle Voyelles et Interactions Bas Niveau}
		Ce modèle proposé par Y. Demazeau \cite{demazeau_interactions_1995} analyse les modèles multi-agents sous 4 points de vues : "\textbf{A}gents", "\textbf{E}nvironnement", "\textbf{I}nteractions" et "\textbf{O}rganisations" (AEIO). Là où ce modèle nous intéresse dans ce chapitre, c'est lorsque J.Tisseau propose d'y ajouter le "U" d'"\textbf{U}tilisateur" \cite{tisseau_realite_2001}, pour former "AEIOU". Dans cette définition, l'utilisateur intervient à souhait dans la simulation, en prenant le contrôle d'un agent, utilisant ses capacités motrices et surchargeant son module de décision. Les autres agents de la simulation réagissent alors à l'utilisateur comme s'il était toujours un agent tout à fait normal. 
		L'utilisateur possède ainsi, à volonté, un "avatar" dans la simulation, interagissant tel un émissaire virtuel avec l'environnement, avec l'ensemble des autres agents et autres systèmes modélisés. Il interagit alors avec les agents de la simulation de manière "horizontale".
		
		
	\subsubsection{Jeux Vidéo et Interactions Haut Niveau}		
		À un autre niveau, certain jeux vidéo de gestion proposent à leur manière un mode d'interaction avec un système multi-agents. Tel "\textit{Populous}", "\textit{Dwarf Fortress}" ou "\textit{Banished}", les "\textit{god games}" et certains jeux de gestions proposent des simulations complexes dont les agents et l'environnement ne sont pas directement contrôlables. L'utilisateur (ici le joueur) interagit avec la simulation à travers des ordres de hauts niveaux, que les agents vont interpréter puis réaliser en suivant leurs propres règles (d'où la dénomination "god game", le joueur agit comme le "dieu" de la simulation). Par exemple, le joueur peut demander la construction d'un certain bâtiment à un endroit précis. Les agents disponibles, s'il y en a, se mettront à la tâche, s'interrompant s'ils en ont besoin sans que le joueur ne puisse intervenir. L'utilisateur n'est alors pas \textit{dans} la simulation, il n'agit qu'en "guidant" les agents, plaçant des tâches dans une "pile", que les agents exécuterons à leur rythme. L'utilisateur interagit ainsi avec les agents de manières "verticales", il dirige, donne un but.
		
		\subsubsection{Modification hors run}
	
	
	\section{Visualisation et Interactions en Environnements Immersifs}	
	
	Nous élargissons désormais nos recherches aux environnements immersifs en général, sans applications directes aux systèmes multi-agents. Ces travaux serviront de base à notre réflexion, couplée à ce que nous avons déjà vu concernant les interactions et visualisations avec des modèles multi-agents.
	
	\subsection{Visualisation en Environnements Immersifs}
	
	Les trois dimensions offertes par les environnements immersifs font de la position dans l'espace la manière principale d'afficher des données. Placées par rapport à un autre objet, comme des unités représentées sur une carte \cite{durbin_battlefield_1998} ou des pathologie sur un corps \cite{coffey_interactive_2012}, la position des données permet une lecture rapide et efficace. Ces données sont alors représentées elles même par un modèle 3D, qui lui peut lui aussi ajouter un niveau d'information sur leur nature. Certains auteurs utilisent même du son pour représenter l'intensité d'un paramètre, comme la métaphore du compteur Geiger \cite{frohlich_exploring_1999}.
	
	
		
		Les travaux connexes sur la visualisation des colonies d'abeilles visent principalement à aider les apiculteurs à prendre des décisions, en leur fournissant des informations sur les populations d'abeilles \cite{engelke_visual_2016, engelke_melissar_2016, nguyen_augmented_2017}. Par exemple, Engelke et. al. \cite{engelke_visual_2016} utilisent la réalité augmentée pour afficher des données provenant de plusieurs capteurs dans une série de ruches réelles et permettre à l'utilisateur de parcourir toutes les données de manière intuitive et immersive. Ces données sont collectées au niveau de la ruche (macro), comme la température et le poids. Ils sont ainsi capables de savoir dans quelle ruche se trouvent certaines abeilles et d'étudier la "dérive des abeilles" : lorsque les abeilles d'une colonie partent et rejoignent une autre colonie. Nous, en revanche, travaillons sur une colonie d'abeilles simulée et nous concentrons sur le "micro" monde, au niveau de l'abeille. Notre objectif n'est pas d'aider (directement) les apiculteurs à prendre des décisions, mais de leur permettre de visualiser et de comprendre les effets de leurs actions sur l'organisation de la colonie (virtuelle), à l'intérieur de la ruche.
	
	\subsubsection{Visualiser une Grande Quantité de Données}
	
	Les nuages de points sont souvent utilisés afin de visualiser une grande quantité de donnée. En plus des 3 axes que nous offrent les trois dimensions de l'environnement immersif, la forme de nos points, leur taille, leur couleur et transparence, permettent de rendre compte d'un grand nombre de variables. De cette manière, Donalek et al. \cite{donalek_immersive_2014} parviennent à afficher 8 dimensions de leurs données. Ces visualisations ont pour but de permettre à l'utilisateur d'observer des relations entre différents éléments, d'observer des schémas et/ou de détecter des données inhabituelles. Plus globalement, elles permettent d'aider l'utilisateur à extraire d'un gros volume de données une information pertinente sur un sous-ensemble de ces données. Un nouvel axe peut permettre de réaliser des animations, permettant d'observer la répartition et les relations entre points changer, en fonction du temps par exemple.
	
	Il est toujours possible de traiter les données avant visualisation : lisser du bruit, agréger des données qui sont peu pertinentes seules, réaliser des opérations pour rendre des écarts plus visibles et bien d'autres.
	
	\subsection{Interactions en Environnements Immersifs}			
	\section*{Conclusion}