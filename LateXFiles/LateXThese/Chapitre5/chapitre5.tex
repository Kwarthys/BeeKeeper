\chapter{Etat de l'art: Simulation Multi-Agents et Environnements Immersifs}

	Dans ce chapitre nous présentons différentes approches utilisées afin d'interagir avec des systèmes multi-agents, ainsi que différentes approches pour les en visualiser différentes propriétés. Nous étendons ensuite nos observations à l'interaction et la visualisation en ce qui concerne les environnement immersifs, où l'utilisateur est plongé au cœur de l'application, souvent grâce à un casque de réalité virtuelle.

	\section{Visualiser et Interagir avec une Simulation Multi-Agents}
		\subsection{Simuler un Système pour Améliorer notre Compréhension}
		
		Simuler un système permet de mieux en comprendre les différentes propriétés. Tel un dessinateur qui se voit obligé d'analyser la moindre forme de son modèle, la simulation (et/ou la modélisation) force l'auteur à se pencher sur le moindre de ses mécanismes. Le modèle, une fois mis en place, peut alors être confronté à la réalité, souvent bien plus complexe, plus subtile.
		
		Ensuite, cette modélisation est un excellent outil pour vulgariser les concepts qui ont du être assimilés, voire créés, pour la concevoir. La modélisation peut aussi permettre d'en manipuler les paramètres, afin d'anticiper le comportement de son homologue réel.
		
		Jacques Tisseau
		\subsection{Visualisation de Systèmes Multi-Agents}
		
		Les travaux connexes sur la visualisation des colonies d'abeilles visent principalement à aider les apiculteurs à prendre des décisions, en leur fournissant des informations sur les populations d'abeilles \cite{engelke_visual_2016, engelke_melissar_2016, nguyen_augmented_2017}. Par exemple, Engelke et. al. \cite{engelke_visual_2016} utilisent la réalité augmentée pour afficher des données provenant de plusieurs capteurs dans une série de ruches réelles et permettre à l'utilisateur de parcourir toutes les données de manière intuitive et immersive. Ces données sont collectées au niveau de la ruche (macro), comme la température et le poids. Ils sont ainsi capables de savoir dans quelle ruche se trouvent certaines abeilles et d'étudier la "dérive des abeilles" : lorsque les abeilles d'une colonie partent et rejoignent une autre colonie. Nous, en revanche, travaillons sur une colonie d'abeilles simulée et nous concentrons sur le "micro" monde, au niveau de l'abeille. Notre objectif n'est pas d'aider (directement) les apiculteurs à prendre des décisions, mais de leur permettre de visualiser et de comprendre les effets de leurs actions sur l'organisation de la colonie (virtuelle), à l'intérieur de la ruche.
		
		Louloudi et. al. \cite{louloudi_new_2012} ont proposé dans leurs travaux de séparer la partie simulation multi-agents de la partie visualisation de leur programme. Séparation offre de pouvoir faire tourner le modèle pleine balle sans la visu quand pas besoin, et visu quand nécessaire. Ils proposent un composant logiciel "relai" entre simu et visu agissant comme un convertisseur universel. Or, les granularité de temps et d'espaces peuvent varier fortement d'un modèle multi agent à un autre, ce qui, couplé à un besoin de transmettre des informations depuis l'application de visualisation vers le simulateur rendent cette notion de relai universel bien difficile à mettre en place.
		
		
		\subsection{Interactions avec des Systèmes Multi-Agents}
		Systèmes en général non immersif, on s'arrête à la RA, et [de ce que j'ai vu], sans utilisation d'interacteurs tangibles.
	Comment mieux comprendre un système complexe : Environnement immersif et interacteurs tangibles. Ruche, cadre et connaissances apicoles.
	
	
	\section{Visualisation en Environnements Immersifs}
	
	
	
	
	
	\section{Interactions en Environnements Immersifs}			
	\section*{Conclusion}