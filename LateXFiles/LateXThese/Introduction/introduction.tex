\chapter*{Introduction}
\addcontentsline{toc}{chapter}{Introduction}
\chaptermark{Introduction}


	Observés depuis de nombreuses années, les insectes sociaux fascinent par leur organisation. Que ce soit les autoroutes de fourmis, les danses des abeilles où les imposantes structure des termites, tout ceci est possible malgré l'absence de hiérarchie, de contrôle central. Nous nous interessons ici principalement aux abeilles domestique \textit{Apis Mellifera}, l'abeille à miel européenne, largement utilisée en apiculture à travers le monde. En effet, ces travaux s'inscrivent dans le projet \textit{SIMBACA}\footnote{SIMBACA : https://siia.univ-brest.fr/simbaca/}, visant à produire une simulation fidèle de ces abeilles a miel, dans un objectif double. Un objectif scientifique se concentrant sur la compréhension des mécanismes complexes régissant la colonie, ainsi que sur les possibilités de tirer de ces simulations des concepts utilisables dans d'autres domaines, ou encore de fournir aux biologistes une plateforme de simulation permettant d'évaluer l'impact de différents stress, parasites, climat, pesticide, sur la colonie, ainsi que les solutions proposées. Le deuxième objectif est pédagogique, se concentrant sur la transmission de connaissance sur la complexité de la colonie, ainsi que l'assistance à la formation de nouveaux apiculteurs, en offrant une simulation fidèle permettant d'itérer plus rapidement et avec moins de risques pour les abeilles sur les différentes procédures apicoles. En effet, là où une action apicole peut nécessiter des mois avant que ses conséquences soit visibles, une simulation permettant d'en observer les conséquences en quelques minutes permetterai de raccourcir les itérations, permettant aux apprenants d'augmenter le nombre d'essai, mais aussi de plus facilement faire le lien entre ce qu'ils ont bien fait, et ce qu'ils auraient pu mieux faire (le tout sans mettre en danger de colonie réelle, alors que le secteur apicole dans le monde entier est en crise\cite{johnson_honey_2010}).

	Le premier objectif de ces travaux est de proposer une simulation de colonie d'abeilles se focalisant sur les différentes tâches à réaliser à l'intérieur de la ruche, et les mécanismes permettant à chaque individu de savoir quelle tâche réaliser : la répartition des tâches (en effet, là où beaucoup de travaux se concentrent sur les mécanismes de butinages, moins ont modélisés l'intérieur de la colonie). Nous utiliserons pour ceci une simulation à base d'agents, nous permettant de nous focaliser sur chaque individu et ses interactions avec les autres individus, et de retrouver les propriétés générales des colonies d'abeilles sans avoir à les préciser : en effet, le comportement simple de chaque individu doit faire émerger un comportement complexe à l'échelle de la colonie. 
	
	Ensuite, le second objectif est d'offrir des moyens de visualisations et d'interactions avec cette simulation, offrant à un utilisateur le moyen d'interagir naturellement avec la simulation, d'altérer son court et d'en observer les différentes propriétés, autant à l'échelle de l'individu que du système complet. Nous désirons ainsi mettre en place une visualisation claire des différents mécanismes permettant la répartition des tâches à l'intérieur de la colonie.
	
	\paragraph{}
	Nous présentons ainsi nos travaux dans ce document. Nous commençons par un chapitre de contexte détaillant les notions de systèmes complexes et celles en découlant, ainsi qu'un état de l'art simplifié sur les connaissances biologiques actuelles des abeilles, de leur physiologie et comportements.
	 Ensuite, le Chapitre I nous permet de réaliser un état de l'art sur les simulations multi-agents, les différents modèles permettant la répartition des tâches, et un tour d'horizon sur des travaux proches présents dans la littérature. 
	 Nous aborderons ensuite notre proposition de modèle de sélection de tâche, basé sur les Modèles à seuils détaillé dans l'état de l'art, et modifié afin d'inclure un plus grand nombre de type de tâches. Nous y développons aussi notre modèle de tâche, divisées en Activité et Actions afin de permettre une grande modularité. 
	 Le Chapitre III décrit l'implémentation de notre simulation à base d'agents de la colonie d'abeilles. Nous y décrivons notre architecture logicielle, la gestion des différents threads ainsi que notre modèle simplifié de la physiologie d'une abeille adulte. 
	 Le Chapitre suivant décrit les différentes hypothèses et expériences mise en place afin de valider ces hypothèses, afin de valider ou non le modèle de sélection de tâche et la simulation en découlant. Nous y indiquons nos résultats puis les discutons, avant de présenter quelques perspectives à la suite de ces travaux. 
	 Nous abordons ensuite un nouvel état de l'art concernant la visualisation et les interactions avec des systèmes multi-agents, que ça soit en environnement immersif ou non, afin de servir de base pour le chapitre suivant. 
	 Dans ce dernier, nous décrivons nos propositions de visualisation et d'interactions dans un environnement virtuel et immersif, respectivement à l'aide d'un graphique en trois dimensions, et en utilisant des interacteurs tangibles : manipuler de réelle cadres (vides) traqués par la simulation qui fera les modifications en conséquences. Nous décrivons dans ce même Chapitre notre évaluation de ces propositions. Un atelier a été mis en place en collaboration avec le Groupement de Défense Sanitaire Apicole du Finistère (\textit{GDSA29}), où quinze apiculteurs ont pu venir réaliser une procédure apicole en réalité virtuelle, et en observer les conséquences simulées. Nous discutons des résultats et présentons quelques perspectives, pour les expérimentations comme pour nos propositions. 
	 Enfin, nous récapitulerons nos propositions, résultats et perspectives dans le chapitre du conclusion, fermant ce document.