\chapter*{Introduction}
\addcontentsline{toc}{chapter}{Introduction}
\chaptermark{Introduction}


	Observés depuis de nombreuses années, les insectes sociaux fascinent par leur organisation. Que ce soit les "autoroutes" de fourmis entre le nid et les ressources, les danses des abeilles ou les imposantes structures des termites, tous ces phénomènes apparaissent malgré l'absence de hiérarchie, de contrôle central. Nous nous intéressons ici principalement aux abeilles domestiques \textit{Apis Mellifera}, l'abeille à miel européenne, largement utilisée en apiculture à travers le monde. 
	En effet, ces travaux s'inscrivent dans le projet \textit{SIMBACA}\footnote{SIMBACA : https://siia.univ-brest.fr/simbaca/}, visant à produire une simulation fidèle de la vie d'une colonie d'abeilles à miel et les phénomènes complexes associés, dans un objectif double. Un objectif scientifique se concentrant sur la compréhension des mécanismes complexes régissant la colonie, dans le but de tirer de ces simulations des concepts utilisables dans d'autres domaines, mais aussi de fournir aux biologistes une plateforme de simulation permettant d'évaluer l'impact de différents stress, parasites, climat, pesticides, sur la colonie, et de tester les solutions proposées. 
	Le deuxième objectif est pédagogique, se concentrant sur la transmission de connaissances sur la complexité de la colonie et de ses mécanismes, ainsi que l'assistance à la formation de nouveaux apiculteurs, en offrant une simulation fidèle permettant d'itérer plus rapidement et avec moins de risques pour les abeilles sur les différentes procédures apicoles. En effet, là où une action apicole peut nécessiter des mois avant que ses conséquences soient visibles, une simulation permettant d'en observer les conséquences en quelques minutes permet de raccourcir les cycles d'apprentissage, augmentant le nombre d'essais des apprenants, mais aussi de plus facilement faire le lien entre leurs actions et leurs conséquences (le tout sans mettre en danger de colonie réelle, alors que le secteur apicole dans le monde entier est en crise \cite{johnson_honey_2010}). Dans le cadre du projet SIMBACA nous travaillons en collaboration avec des biologistes de l'unité "Abeilles et Environnement" de l'\textit{Institut National de Recherche pour l'Agriculture, l'Alimentation et l'Environnement} (INRAE), ainsi qu'avec le \textit{Groupement de Défense Sanitaire Apicole du Finistère} (GDSA29).

	Le premier objectif de ce travail de thèse est de proposer une simulation de colonie d'abeilles se focalisant sur les différentes tâches à réaliser à l'intérieur de la ruche, et les mécanismes permettant à chaque individu de savoir quelle tâche réaliser : la répartition des tâches. Nous utiliserons pour ceci une simulation à base d'agents, nous permettant de nous focaliser sur chaque individu et ses interactions avec les autres individus, et de retrouver les propriétés générales des colonies d'abeilles sans avoir à les préciser : en effet, le comportement simple de chaque individu doit faire émerger un comportement complexe à l'échelle de la colonie. 
	
	Ensuite, le second objectif est d'offrir des moyens de visualisations et d'interactions avec cette simulation, permettant à un utilisateur d'interagir naturellement avec la simulation, d'altérer son cours et d'en observer les différentes propriétés, autant à l'échelle de l'individu que du système complet. Nous désirons ainsi permettre à un utilisateur de comprendre par une visualisation claire les différents mécanismes permettant la répartition des tâches à l'intérieur de la colonie.
	
	\paragraph{}		
		Modéliser et simuler un système permet de mieux en comprendre les différentes propriétés, leurs interactions et met en valeur les aspects encore inconnus de ce système. Tel un dessinateur qui se voit obligé d'analyser la moindre forme de son modèle, la simulation (et/ou la modélisation) force l'auteur à se pencher sur le moindre de ses mécanismes. Le modèle, une fois mis en place, peut alors être confronté à la réalité, souvent bien plus complexe, plus subtile. Ensuite, cette modélisation est un excellent outil pour vulgariser les concepts qui ont dû être assimilés, voire créés, pour la concevoir. La modélisation peut aussi permettre d'en manipuler les paramètres, afin d'anticiper le comportement de son homologue réel.
		
	\boitemagique{Jacques Tisseau \cite{tisseau_realite_2001}}{"\textit{Les fonctions d'un modèle sont multiples. Pour le modélisateur, le modèle permet d'imaginer, de concevoir, de prévoir et d’améliorer sa représentation du phénomène ou de l'idée qu'il cherche à modéliser. Le modèle devient un support de communication pour représenter, sensibiliser, expliquer ou enseigner les concepts concernés. À l'autre bout de la chaîne, le modèle aide l'utilisateur dans sa compréhension du phénomène (ou de l'idée) représenté(e); il peut également l'évaluer et l'expérimenter par simulation.}"}
	
	Le projet \textit{SIMBACA} s'inscrit dans l'ensemble de cette démarche, de la construction du modèle de la colonie à la vulgarisation aux apiculteurs, en passant par la production d'un outil pour les biologistes de l'abeille.
	
	\paragraph{}
	Dans ce document, nous commençons par un chapitre de contexte détaillant les notions de systèmes complexes et celles en découlant, ainsi qu'un état de l'art simplifié sur les connaissances biologiques actuelles des abeilles, de leur physiologie et comportements.
	 Ensuite, le \textit{Chapitre \ref{ChapitreEASMA}} nous permet de réaliser un état de l'art sur les simulations multi-agents, les différents modèles permettant la répartition des tâches, et un tour d'horizon sur les principaux travaux présents dans la littérature s'intéressant de près ou de loin aux insectes sociaux et leur auto-organisation. 
	 Nous aborderons ensuite dans le \textit{Chapitre \ref{ChapitrePropDecision}} notre proposition de modèle de sélection de tâches, basé sur les Modèles à seuils détaillés dans l'état de l'art et modifiés afin d'inclure un plus grand nombre de type de tâches, en utilisant différentes notions de motivation internes à nos agents. Nous y développons aussi notre modèle de tâches, ces dernières divisées en Activités et Actions afin de permettre une grande modularité. 
	 Le \textit{Chapitre \ref{ChapitrePropSMA}} décrit l'implémentation de ce modèle pour la simulation à base d'agents de la colonie d'abeilles. Nous y décrivons notre architecture logicielle, la gestion des différents \textit{threads} ainsi que notre modèle simplifié de la physiologie d'une abeille adulte. 
	 Le chapitre suivant décrit les différentes hypothèses et expériences misent en place afin de valider ces hypothèses, le modèle de sélection de tâches et la simulation en découlant. Nous y indiquons nos résultats puis les discutons, avant de présenter quelques perspectives à la suite de ces travaux. 
	 Nous abordons ensuite dans le \textit{Chapitre \ref{ChapitreEAVR}} un nouvel état de l'art concernant la visualisation et les interactions avec des systèmes multi-agents, que ça soit en environnement immersif ou non, afin de servir de base pour le chapitre suivant. 
	 Dans ce dernier, nous décrivons nos propositions de visualisation et d'interactions dans un environnement virtuel et immersif, respectivement à l'aide d'un graphique en trois dimensions, et en utilisant des interacteurs tangibles : manipuler de réels cadres (vides) traqués par la simulation qui fera les modifications en conséquences. 
	 Nous décrivons dans le \textit{Chapitre \ref{ChapitreEvalVR}} notre évaluation de ces propositions. Pour ceci un atelier a été mis en place en collaboration avec le GDSA29, où quinze apiculteurs ont pu venir réaliser une procédure apicole en réalité virtuelle, et en observer les conséquences simulées. Nous discutons des résultats et présentons quelques perspectives, pour les expérimentations comme pour nos propositions. 
	 Enfin, nous récapitulerons nos propositions, résultats et perspectives dans le chapitre de conclusion, fermant ce document.