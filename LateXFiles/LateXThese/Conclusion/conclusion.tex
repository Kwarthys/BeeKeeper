\chapter*{Conclusion et Perspectives}
\addcontentsline{toc}{chapter}{Conclusion et Perspectives}
\chaptermark{Conclusion et Perspectives}

Il est désormais temps de revenir sur les différents travaux et notions abordés pendant cette thèse, fortement marquée par la pluridisciplinarité. Nous parlons en effet aussi bien de la biologie de l'abeille domestique que d'interactions tangibles en environnement immersifs, en passant par les algorithmes de prises de décisions pour des systèmes multi-agents.

Dans ces travaux nous avons proposé une simulation de colonie d'abeilles interactive, permettant à l'utilisateur de réaliser quelques actions apicoles classiques en environnement immersif et à l'aide d'un interacteur tangible.

%Ch1
Nous avons présenté les notions de systèmes complexes, émergence, chaos, boucles de rétroactions et auto-organisation. Nous avons ensuite présenté un état de l'art sur les connaissances biologiques actuelles de l'abeille domestique, et les nombreuses hypothèses toujours présentes autour de leur comportement. Par cette analyse nous mettons en valeur les différentes propriétés d'une colonie d'abeilles en faisant un système complexe. Ces données nous ont permit par la suite de construire un modèle informatique simplifié d'une abeille adulte, de sa physiologie à ses différentes prises de décisions (et les mécanismes nécessaires pour ces prises de décisions).

%Ch2
Nous avons présenté un état de l'art concernant la prise de décisions d'agents réactifs dans des simulations à base d'agents, afin de trouver un modèle nous permettant de modéliser une colonie d'abeilles. Cet état de l'art nous permet d'affirmer que les modèles à seuils semblent adaptés, mais ne permettent pas en l'état d'inclure un type de tâche dont nous avons besoin : les tâches ne présentant pas de stimulus déclencheur évident (que nous retrouvons chez les abeilles). Nous avons alors proposé de construire une adaptation de modèles à base de seuils, utilisant différentes notions de motivation interne.

%Ch3
La motivation source représentant la volonté de l'agent nous permet de créer un stimulus artificiel pour calculer un score cohérent à une tâche sans stimulus déclencheur évident. Ensuite, la motivation interne que nous appelons motivation "guide" permet à nos agents d'avoir une notion de leur efficacité dans leur tâche actuelle. Elle nous a permis de mettre en place un système d'interruption afin de permettre aux agents de toujours effectuer la tâche la plus prioritaire. Ces deux modifications nous permettent d'utiliser des modèles à seuils pour des tâches ayant un stimulus déclencheur et d'autres n'en ayant pas, de manière cohérente. Nous avons ensuite proposé une modélisation de tâche incluant ces concepts de motivations, permettant aux agents de savoir quel comportement exécuter. Divisé en deux concepts, Activité et Action, nos tâches sont organisées en subsomptions hiérarchiques, nous offrant une grande modularité.

%Ch4
Nous avons présenté notre modèle informatique, tiré de la biologie, de la colonie d'abeilles virtuelle. L'implémentation du simulateur est détaillée de son architecture logicielle à son comportement dynamique, notamment l'utilisation de nombreux \textit{threads} afin de réduire les temps de simulation. Nous avons décrit le processus de calibration expérimentale du modèle de la colonie, rendue nécessaire par la présence de nombreux comportements émergents. Ces calibrations ont majoritairement concerné les quantités et effets des phéromones et hormones régissant la physiologie et les décisions de nos abeilles virtuelles. Nous avons aussi présenté nos deux versions du modèle de la colonie, importantes dans notre démarche incrémentale de la mise en place de ce système complexe. La première version simplifiée, sans pontes, naissances ni morts, visait à valider les premières notions d'auto-organisation, et la "version cycle complet" avait pour objectif de se rapprocher d'une simulation fidèle d'une colonie d'abeilles.

%Ch5
Ainsi ces modèles et leur simulateur ont été évalués suivant deux hypothèses, auto-organisation et cohérence biologique. La première version du modèle est parvenue à reproduire l'auto-organisation mais pas de manière fidèle à la biologie, les processus biologiques étant nettement accélérés afin de valider le modèle de sélection de tâches. La deuxième version quant à elle est parvenue à valider nos deux hypothèses, avec toutefois quelques points à améliorer. La littérature de biologie va nous guider pour la suite de ces travaux, dans les nouvelles versions du modèle de la colonie.

\paragraph{}

%Ch6 & 7
Le système multi-agents et son simulateur désormais validés, nous avons abordé les questions de visualisations et d'interactions avec de tels systèmes dans la littérature. Rapidement, nos recherche s'élargissent dans le domaine de la visualisation et de l'interaction avec des systèmes complexes, qui présentent de nombreux défis. En effet, les différences d'échelles, de niveaux, de granularités temporelles et spatiales rendent ces systèmes difficiles à représenter, et à expliquer. La notion de \textit{macroscope}, un outil pour observer "l'infiniment complexe" et différentes lignes de conduite à ce sujet nous aident à concevoir nos visualisations. Nous avons ainsi proposé un nuage de points réparti sur trois axes, le "Graph3D", permettant de rendre compte de manière abstraite de l'état interne de chaque abeille de la colonie. Présentant la corrélation entre l'âge réel, l'âge physiologique et les échanges de phéromones de chaque agent, il permet d'observer directement l'état interne de la colonie à travers la forme et les couleurs du nuage de points.

Nous effectuons par la même occasion un tour d'horizon concernant les visualisations et interactions en environnement immersifs, plongeant l'utilisateur au centre de la simulation. Cet état de l'art nous a conduit à proposer d'utiliser d'une part un modèle 3D de la ruche et de ses cadres pour permettre à l'utilisateur d'avoir une vision immersive et "réaliste" de la colonie, et d'autre part des objets tangibles pour faciliter les interactions avec cette ruche virtuelle, car les interactions tangibles présentent de nombreux avantages didactiques.

%Ch8
Une fois décrites, ces propositions ont été évaluées lors d'une expérimentation réalisée avec quinze apiculteurs du \textit{GDSA29}. Pour des raisons matérielles, sanitaires et des contraintes de temps, l'interacteur tangible ne faisait pas partie de cette expérimentation. Nous avons pu confronter le Graph3D et l'environnement immersif à la réalité des apiculteurs, et avons obtenus de précieux retours. Ils ont en effet confirmé l'intérêt pédagogique d'une interaction avec une simulation de colonie d'abeilles en environnement immersif. Le Graph3D quant à lui n'a pas rempli son rôle. En effet, nous l'avons créé dans le but d'obtenir une \textit{visualisation} des mécanismes régissant la répartition des tâches au sein de la colonie, mais l'avons évalué en tant qu'outil d'\textit{enseignement} de ces mécanismes. Or, montrer et enseigner sont des actions bien différentes. Ainsi, notre moyen de visualisation proposé, un outil didactique reste à mettre en place afin d'enseigner et/ou de sensibiliser aux mécanismes complexes régissant la colonie.
	
	 La pluridisciplinarité de nos travaux nous a permis de travailler beaucoup d'aspects différents de nos propositions, mais la contrainte de temps nous a ainsi empêché de tous les développer comme nous le souhaitions. C'est donc naturellement que de nombreuses perspectives s'offrent à nous, dans de nombreux domaines.
	
\section*{Perspectives}
	Nous proposons ici de présenter les différentes perspectives énoncées le long de ce document en trois catégories, des plus courts termes au plus longs termes.
	
	\subsection*{Perspectives Courts Termes}	
	Nous avons présenté dans les perspectives du \textit{Chapitre \ref{ChapitreEvalSMA}} une nouvelle version du modèle de prise de décision pour nos agents réactifs à base de seuils et de motivations internes. Cette implémentation nous permettrait de nous passer d'un mécanisme d'interruption dédié, en combinant les deux types de motivation directement lors du calcul du score de chaque tâche, propre aux systèmes à seuils. Ainsi cet algorithme s'intègrera plus naturellement dans celui des systèmes à seuils, ne venant influer que sur un stimulus artificiel. L'implémentation de cet algorithme pourra être appliquée à des problèmes classiques du domaine, afin d'en tester les performances, pouvant donner lieu à une publication. Sa relative simplicité et son intégration plus douce promettent des résultats intéressants. Son intégration au modèle multi-agents de la colonie pourrait quant à elle nécessiter plus de temps, sans qu'il soit simple d'anticiper à quel point, et quels en seront les impacts sur le comportement des agents (supposément minimes, il est toutefois important de se montrer prudent en présence de systèmes complexes et d'un potentiel "effet papillon").
	
	Une nouvelle version du Graph3D, en tant qu'outil de visualisation et non plus de vulgarisation, présentée dans le chapitre précédent, pourrait aussi être réalisée dans un temps relativement court. L'ajout du "couloir" représentant les trajectoires classiques des agents respectant le polyéthisme d'âge dans la visualisation aiderait à détecter les instances où l'auto-organisation provoque de profonds changement dans la démographie de la colonie.
	
	\subsection*{Perspectives Moyens Termes}	
	Difficile de dire si cette perspective tient du moyen ou du long terme, mais le développement d'un outil pédagogique à base du Graph3D, visant à vulgariser, mettre en valeur et expliquer les différents mécanismes à l'œuvre dans l'auto-organisation de la colonie serait tout à fait intéressant. De l'ordre du scénario didactique, proche d'un "\textit{serious game}", cette application, immersive ou non, pourrait aussi utiliser les interactions tangibles que nous avons explorées pour améliorer l'expérience pédagogique.
	
	\subsection*{Perspectives Longs Termes}
	Le modèle multi-agents de la colonie présente à l'heure actuelle de nombreuses simplifications. Comme énoncé dans le \textit{Chapitre \ref{ChapitreEvalSMA}}, la nourriture y est au plus simple. 
	
	Intégrer une simulation plus poussée des flux de nourriture serait un immense atout pour la simulation et les possibilités alors offertes. Cependant, ce simple ajout a des répercussions sur l'ensemble du système et nécessite alors un travail beaucoup plus élaboré. 
	
	Il sera en effet nécessaire d'ajouter dans la simulation le rôle des "receveuses", clé dans l'auto-organisation de la colonie. Le butinage ainsi que les quantités de nourritures rapportées, consommées et entreposées devront être tirées de la biologie. Le comportement des butineuses, des receveuses et des nourrices devront alors être revus afin d'intégrer ces changements.
	
	Il sera sûrement nécessaire d'introduire une nouvelle phéromone émise par le couvain, l'Ocimène, guidant les nourrices vers les larves ayant besoin de nourriture (elles se déplacent pour l'instant aléatoirement). Ce guidage améliorera les capacités des nourrices virtuelles à s'occuper du couvain. Nous pourrions ainsi ajouter un temps de "préparation" à la nourriture du couvain, que les nourrices préparent en réalité avec leurs glandes et qui nécessite du temps. Ces changements nous rapprocheront du fonctionnement et des performances des nourrices réelles.
	
	Un butinage plus poussé nous offrira l'opportunité d'intégrer un module pour l'environnement extérieur, nous permettant alors de simuler différentes floraisons et conditions climatiques, impactant la collecte de ressource de la colonie. De plus, une fois cette chaine de la nourriture mise en place, nous pourrions alors simuler des attaques du parasite "Nosema". Il sera aussi possible de simuler des attaques de l'acarien "Varroa", et son impact sur la colonie. Si les flux de nourritures n'ont pas l'air nécessaire à son intégration, ce parasite affaiblit les butineuses, et c'est en ça que ce flux doit être présent pour obtenir des résultats intéressants.
	
	Tout est lié, cette simple propriété des systèmes complexes rend difficile l'ajout de "petites" modifications. Toutes ces "petites" modifications nécessitent la plupart du temps de nombreux autres "petits" ajouts, rendant rapidement la mise à jour conséquente.
	