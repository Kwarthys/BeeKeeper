\chapter*{Contexte : Colonie d'abeilles et Complexité, Biologie et Système complexe}
\addcontentsline{toc}{chapter}{Contexte : Colonie d'abeilles et Complexité, Biologie et Système complexe}
\chaptermark{Contexte : Colonie d'abeilles et Complexité, Biologie et Système complexe}
	\label{sectionBio}
	
	Les insectes sociaux sont depuis longtemps étudiés pour leurs capacités complexes à se répartir dynamiquement le travail, sans l'aide d'un contrôle central. Ce chapitre présente tout d'abord les principales notions de complexité, et donne plusieurs exemples de phénomènes complexes que l'on peut retrouver dans la vie d'une colonie d'abeilles. Nous verrons dans le chapitre suivant quelques modèles Multi-Agents présents dans la littérature servant à modéliser ces systèmes.

	
		\subsubsection{Systèmes complexes}
		
			Il n'existe à ce jour pas de définition précise de ce qu'est un système complexe \cite{heylighen_complexity_2008}, du fait de l'étendue vertigineuse des domaines et cas d'application touchés. Nous pouvons cependant parler de système complexe pour ceux qui suivent l'adage "Le tout vaut plus que la somme des parties" \cite{edmonds_what_1999}. En effet, nous retrouvons souvent dans la littérature les notions de chaos, d'interactions locales et d'imprévisibilité. Au centre de tout ceci ce trouvent les boucles de rétroactions. Les boucles de rétroaction se trouvent sous deux formes, les rétroactions positives et négatives. Les boucles de rétroactions positives concernent moins les systèmes complexes, car elles n'ont pas de rôle de régulation, mais plutôt d'accélération. Une bombe nucléaire fonctionne avec ce principe, une fission va en déclencher plusieurs autres, qui vont à leurs tours en déclencher bien d'autres, sous une forme d'escalade de la violence exponentielle. Sans autre mécanisme pour les réguler, ce sont des phénomènes très transitoires.
			
			Nous nous intéressons ici surtout aux boucles de rétroactions négatives. Ces boucles consistent en mécanismes régulateurs, stabilisant un système vers un équilibre. Nous pouvons en trouver de toutes sortes, sous bien des formes. Le soleil par exemple, voit sa forme maintenue par ce qui est appelé un équilibre hydrostatique : lorsque son cœur réalise la fusion nucléaire, il crée une immense force au centre du soleil le faisant s'étendre. Celui-ci gonfle alors et fait ainsi baisser la pression en son centre, le nombre de fusion est alors légèrement réduit, la force poussant le soleil à s'étendre est alors contrée par la gravité, le poussant de toute part vers son centre. Le soleil s'effondre alors sur lui même ! Mais du même temps, la pression en son centre remonte, les fusions reprennent de plus belle et il gonfle à nouveau : le soleil danse alors légèrement autour de son point d'équilibre, ce qui peut être vu comme une boucle de rétroaction. Autre exemple bien différent, un régulateur de vitesse de voiture ! Lorsque la vitesse mesurée est trop basse, le régulateur déclenche une accélération, qui augmente alors la vitesse du véhicule. Le régulateur arrête d'accélérer lorsque la vitesse est satisfaisante. Le véhicule est alors ralenti par les différents frottements sur ses composants, jusqu'à ce que la vitesse soit à nouveau assez basse pour déclencher une nouvelle accélération.
			
			De cette manière, nous pouvons imaginer un système peuplé de nombreuses entités aux caractéristiques bien différentes, mais qui interagissent entre-elles via diverses boucles de rétroactions. C'est ainsi que l'ordre semble émerger du chaos, nous sommes alors faces à un système complexe ! Un bel exemple de tels systèmes auxquels nous sommes soumis tous les jours est le climat. De faibles interactions qui peuvent paraitre insignifiantes ont de grosses répercussions sur le système dans son ensemble. Nous augmentons d'un demi degré la température des océans et ce sont les courants marins, les vents et tempêtes, les chaines alimentaires et bien d'autres qui se dérèglent. Augmenter de quelques points le pourcentage de gaz à effet de serre dans l'atmosphère et le circuit s'emballe et réchauffe le tout, augmentant la fréquence d'événements violents etc \cite{allen_2018_2018}. Une grande quantité d'entités, jusqu'à l'échelle moléculaire, vont interagir entre elles, faisant varier leurs températures, leurs pressions, leurs absorptions et réflexions de la lumière altérant par la même occasion le climat sur une échelle planétaire, affectant jusqu'à notre mode de vie. 
						
			Nous allons maintenant constater en quoi une colonie d'abeilles est un bon représentant de la grande famille des systèmes complexes.
			
			

	
		\subsubsection{Abeilles et Systèmes complexes}		
			Une colonie classique d'abeilles domestiques \textit{Apis Melifera} héberge en moyenne quelques dizaines de milliers d'individus, 50 000 est un nombre qui revient souvent. Tous ces individus ont des besoins en nourritures et en eau, mais l'attention au couvain, regroupant tous les stades de vie de l'abeille avant sa phase adulte, demande un tout autre niveau d'attention. Nourriture spéciale, température précise, des cellules de cires, et nettoyées en plus ! Tous ces besoins créent une chaine logistique impressionnante, que des apiculteurs ou chercheurs observent depuis des centaines d'années. En effet, prouesses remarquable, les abeilles (et bien d'autres insectes sociaux) arrivent à survivre et même à s'épanouir sans aucun contrôle central, aucun individus spéciaux chargés du management ou de la surveillance des stocks. Des rôles ont été créés pour classer les différents maillons de cette chaîne, dont nous allons décrire les principaux \cite{winston_biology_1991, winston_role_1991, seeley_age_1991}. 
			
			Emblématique de la colonie d'abeilles et présentant des différences physiques notables, la Reine, plus grande que les autres abeilles, est chargée de pondre à une vitesse ahurissante : près d'un œuf par minute en moyenne sur toute sa vie. Elle est en permanence entourée d'une "cour royale", d'autres abeilles qui viennent la lécher de tout côté, attirées par ses phéromones puissantes. Elles vont ensuite au fil de leur vie, peut être même sans s'en rendre compte, répandre les phéromones royales dans toutes la colonie, assurant ainsi la suprématie de la reine. Souvent ces abeilles sont des nourrices, elles sont chargées de nourrir le couvain. De vraies cuisinières, elles parcourent la ruche à la recherche de miel et de pollen afin de concocter un liquide calorifique dont les larves se nourrissent. Parfois, une recette alternative encore plus riche, la célèbre gelée royale, est donnée à des larves spécialement sélectionnées pour devenir de nouvelles reines.
			
			Les larves ne sont que d'énorme systèmes digestifs autonomes. Elles ingurgitent des quantités gigantesque de nourriture comparé a le poids, pour assurer leur croissance rapide, et le permettre de devenir par la suite de solides ouvrières. La reine pond des œufs, qu'elle dépose au fond de cellules. Ces cellules doivent être bâties par des ouvrières cirières, et nettoyée, la reine ne pond jamais dans une cellule sale. Une fois pondu, un œuf va mettre trois jours pour évoluer en larve, puis consommer son mélange de miel et de pollen pendant les six jours que représentent l'état de larve, puis va devenir une nymphe, étape importante où la larve va se métamorphoser progressivement en sa forme finale, une abeille adulte. Dès qu'une larve passe au stade de nymphe, elle est repérée par une ouvrière cirière qui va méthodiquement operculer la cellule : créer une sorte de toit, celant la nymphe qui devra ouvrir sa cellule elle même, dans ses premières heures d'adulte, à l'aide de ses nouvelles mandibules.
			
			Pour que toute cette croissance se passe bien, la température du couvain doit être exactement de 35°C. Un écart de l'ordre du demi degré peut engendrer des pertes colossales, et mettre la colonie en péril. Lorsqu'il fait légèrement trop chaud nous trouvons des abeilles thermo-régulatrice : elles battent des ailes sur le couvain et/ou au niveau de la sortie de la ruche pour créer un courant d'air et rafraichir la ruche. Certaines vaporisent même de l'eau dans la ruche pour aider a faire chuter la température. Lorsqu'il fait froid, les abeilles se resserrent progressivement au dessus du couvain, afin de créer ce qui est appelé la "grappe" et tenir le couvain au chaud, grâce a leur corps. Si leur simple présence ne suffit plus, elles peuvent actionner certains de leur muscle afin de générer un peu plus de chaleur, ce qui arrive pendant l'hiver, phase critique pour la colonie. Si la grappe est trop petite pour recouvrir l'ensemble de couvain, alors une bonne partie de ce-dernier, celui en périphérie qui n'aura pas été protégé par les adultes, va mourir, on l'appelle alors froidement le "couvain refroidi".
			
			Les abeilles sont surtout célèbres pour leur butinage, rôle tenu par les "butineuses". Celles-ci sélectionnent les meilleurs sites de ressources sur quelques kilomètres à la ronde et ramènent nectar et pollen à la colonie. Certaines abeilles plus téméraires vont spontanément quitter la ruche sans destination, dans le seul but de trouver une nouvelle source de nectar. Ces butineuses sont appelées des "scouts", ou des "éclaireurs". Elles repèrent les fleurs grâce à leurs immenses yeux sensibles aux ultra-violet, que les fleurs ont appris à bien réfléchir dans une co-évolution avec les pollinisateurs. Une fois sur une source de nectar, il est temps d'en juger la qualité. L'abeille va récolter le nectar, et estimer sa teneur en sucre. Si cette teneur lui convient, elle va rentrer à la colonie et commencer la troisième étape, le point clé, le recrutement. Mais avant ceci, une fois rentrée, la butineuse cherche une "receveuse" : une fois trouvée, elle lui donne une partie de sa charge de nectar. Les butineuses donnent ainsi leur récolte à trois ou quatre receveuses. Ensuite, elles viennent déposer sans trop d'attention leurs ballots de pollen dans des cellules proches du couvain, puis repartent. Avant de décoller, certaines dansent ! L'abeille recruteuse va se mettre à danser la désormais célèbre \textit{"Waggle Dance"}, en forme de 8. Cette danse a un objectif double. Le premier : communiquer la position de la source par rapport au soleil, pour permettre aux autres de la retrouver. Le second est plus indirect : plus la source est sucrée et proche, et donc profitable, plus l'abeille va danser longtemps, et même répéter la danse dans plusieurs endroits de la ruche. Une danse plus longue offre plus de temps à d'autres abeilles de venir la suivre et apprendre la position de cette nouvelle source. Ainsi, plus la source est sucrée, plus la recruteuse va communiquer la position à de nombreuses butineuses.
			Dans le même temps, les receveuses vont déposer le nectar reçu dans des cellules, toujours très hautes sur le cadre. Certaines iront aussi tasser les ballots de pollen dans leurs cellules, afin de ne pas perdre de place. Le nectar est alors prêt à subir ses transformations pour devenir du miel. Certaines abeilles vont alors venir "cracher" sur le nectar, afin d'y déposer leurs enzymes qui feront le travail de transformation. Elles y déposent du même coup une substance antiseptique, s'assurant ainsi que le miel ne sera pas contaminé par de mauvaises bactéries ou virus. Ensuite, pendant ce travail d'enzymes, des "ventileuses" viendront se placer au dessus du miel et battre frénétiquement des ailes. Elles ajustent ainsi l'hygrométrie du miel, l'amenant très précisément a 15\% d'humidité, parfait pour la conservation.
			
			Toutes ces ouvrières sont sœurs ou demi-sœurs. Elles ont toutes la même mère, mais pas forcément le même père. Au moment de la ponte, la reine "choisi" de féconder ou non son œuf. Par la magie de la biologie, un œuf non fécondé donnera un mâle, un œuf fécondé donnera une ouvrière. Les mâles ont une vie très particulière : ils sont incapable de quoi que ce soit, ni de se laver, ni de se nourrir seuls, et ne font rien dans la colonie. Dès que la situation se gâte pour la colonie et que les ressources se font rares, ce sont les premiers à être chassés de la colonie. Reconnaissable a leurs immenses yeux, ils ne sont bon qu'à un chose, s'envoler au bon moment vers un point de rencontre défini (on ne sait pas trop comment), localiser une reine et la féconder. Ils meurent instantanément après l'acte (globalement leur intérieur explose). Une reine est ainsi fécondée par une dizaine de mâles et va garder leurs semences dans sa spermathèque et s'en servir tout au long de sa vie.
			
			Un autre mécanisme intéressant à été montré par \cite{maisonnasse_e-b-ocimene_2010}. La présence de couvain augmente la proportion de nourrices dans la colonie, ce qui semble intuitif. En revanche, ils ont montré que la présence de couvain très jeune, moins de 3 jours, augmente fortement l'activité de butinage, et plus précisément de butinage de pollen. Nous allons désormais nous intéresser plus en détails à ces nombreux mécanismes de contrôles comme celui-ci. Comment les abeilles parviennent à savoir quoi faire, sans personne pour diriger le tout.
			
		\subsubsection{Auto-Organisation de la Colonie}
			Afin d'assurer le bon fonctionnement et l'épanouissement de la colonie, chaque individu la composant doit effectuer un certain nombre de tâches lorsqu'elles sont nécessaires, et sans aucun contrôle central. Par exemple, personne pour surveiller la température et assigner une équipe à la régulation de la température pendant un certain temps. Nous parlons alors d'auto-organisation. Chaque individu utilise ses perceptions locales pour savoir quel travail réaliser, ainsi nous parlons aussi d'allocation du travail. Les quelques mécanismes complexes d'auto-organisation (et leurs boucles de rétroactions) que nous allons aborder sont les suivants (il en existe bien d'autres, et surement une majorité dont nous ignorons pour l'instant jusqu'à l'existence) :
			
			
			\begin{itemize}
				\item Thermorégulation basée sur la perception locale de température et sur les différences entre individus.
				\item Sélection des meilleures sources de nourritures par les butineuses lors du recrutement.
				\item Régulation de l'âge du premier butinage en fonction des demandes du couvain.
			\end{itemize}

			\subsubsection{Thermorégulation}			
			Un des points clés de cette organisation est que chaque individu a des "tolérances" légèrement différentes. Une diversité qui provient notamment de la présence des ces fameuses "demi-sœurs", dont nous avons parlé juste avant. J. Jones \cite{jones_honey_2004} et.al. ont montré que l'exceptionnelle capacité des abeilles à réguler si précisément la température est en grande partie lié à cette diversité génétique. En effet, pour que la colonie régule correctement, il est important d'éviter des mouvements de foules à l'échelle de la colonie, afin de ne pas osciller entre trop chaud et trop froid en permanence. La diversité génétique modifie légèrement les seuils des tolérances des abeilles, qui vont alors progressivement réguler la température.
			
			De cette manière, chaque individu mesure et juge la température actuelle de la colonie, et prends alors la décision de réchauffer, refroidir, ou réaliser d'autres devoirs. Sa position dans la ruche et sa sensibilité personnelles vont entrer en compte dans cette décision. Ainsi, par l'action autonome de chacun des individus, la température du couvain ne s'écarte jamais plus d'un degré de 35°C.
			
			\subsubsection{Sélection des meilleures sources de nectar}
			Lors de la collecte du nectar, les butineuses sont très sélectives et préfèrent les nectars aux hautes teneurs en sucres sur des fleurs placés le plus proche possible de la ruche. La colonie utilise donc un système de recrutement, presque de recommandation, afin d'allouer ses effectifs de butineuses de manière optimale et dynamique, préférant les sources proches et nutritives.
			Une fois recrutées, les nouvelles butineuses vont se rendre à la source et répéter le processus : collecter, juger, rentrer et parfois recruter. Là encore, la diversité est clé : des abeilles moins portées sur la communication vont passer moins de temps à recruter, afin de maximiser le temps de butinage. En effet, il y a tout de même des dizaines de milliers de bouches à nourrir ! De plus, certaines butineuses sont moins difficiles que d'autres, elles vont donc communiquer des sources de faible qualité et y maintenir un faible contingent. Ce contingent alors moins utile dans l'immédiat sert de surveillance, car les teneurs en sucres des différents nectars peuvent fortement varier selon les saisons mais aussi pendant les périodes de la journée. Une source de faible qualité peut alors devenir une source extrêmement intéressante en quelques heures. Le groupe alors déjà présent peut observer ce changement et déjà danser dans toute la colonie pour avertir les autres, gagnant ainsi un temps précieux de re-découverte de la source mais aussi le temps d'amorcer une réponse conséquente. Gagner du temps sur une réaction exponentielle est toujours extrêmement précieux. C'est en effet une boucle de rétroactions positive, plus il y a d'abeilles à apprécier la source, plus il y aura de recruteuses, augmentant ainsi encore le nombre de recruteuses par la suite.
			
			la diversité génétique fait ici effet de régulation, et permet d'ajuster le butinage en un savant équilibre entre exploration et exploitation. Il est certes important de ramener d'énorme quantité de nourriture a la colonie, mais avoir la totalité de ses effectifs sur une même source présente des risques. Les abeilles "scouts", ainsi que les "moins difficiles" citées plus tôt vont alors permettre de maintenir l'exploration et la découverte de nouvelle ressources à un niveau sécurisant.
			
		\subsubsection{Phéromones et Physiologie}
			\label{subsubPhysio}
			L'auto-organisation de la colonie ne se joue pas que dans les perceptions, la colonie s'appuie sur des mécanismes indirects, long-termes, physiologiques, qui prennent place grâce à différentes hormones et phéromones.  Nous étudions ici en détail l'importance physiologique des glandes hypopharyngiennes (GH) et de la Corpora Allata. C'est ce mécanisme qui nous intéresse tout particulièrement dans ce manuscrit. La Corpora Allata permet aux adultes de sécréter une hormone appelée Hormone Juvénile(HJ). Cette hormone est retrouvée en grande quantité chez les butineuses, et en faible quantité chez les nourrices. Les GH permettent aux abeilles s'occupant du couvain de transformer le pollen et le nectar en une substance riche destinée aux larves. Elles permettent aussi aux butineuses de traiter chimiquement le nectar, le rendant utilisable pour les nourrices, transformable en miel et même consommable directement par les autres adultes. Or, ces deux comportements sont incompatibles, les GH subissent une modification physiologique pour effectuer l'une ou l'autre de ces fonctions. De plus, les abeilles (ainsi que d'autres insectes sociaux) emploient différentes phéromones pour parvenir à s'épanouir. Ces phéromones peuvent être séparées en deux catégories : les phéromones modificatrices, et les phéromones incitatrices. Les phéromones modificatrices, comme le nom l'indique, viendront modifier la physiologie des individus : ce sont elles qui sont notamment responsable de la modification des propriétés des GH que nous venons d'aborder. Les phéromones incitatrices quant à elles viendront déclencher des comportements, sans altération physique sur les agents. Ainsi, nous pouvons voir les phéromones incitatrices comme ayant un effet très court terme, et les phéromones modificatrices un effet très long terme. Par exemple, lors d'une attaque, les premières abeilles témoins viendront émettre une phéromone d'alarme très volatile, qui aura pour effet de faire sortir une grande quantité d'abeilles de la ruche pour servir de renfort. Cette phéromone incitatrice augmente aussi fortement l'agressivité des abeilles. Fait étonnant, d'après certains apiculteurs, cette phéromone aurait parfois une odeur proche de la banane !
			
			
			
			
			\begin{figure}
				\includegraphics[width=\textwidth]{Pictures/Figures/PhModificatrices.png}
				\caption{Synthèse des connaissances simplifiées et schématisées des phéromones modificatrices au sein de la colonie.}
				\label{phMod}
			\end{figure}	
			
			La Figure \ref{phMod} schématise les interactions entre phéromones modificatrices, hormones et glandes, ainsi qu'entre différents individus de la colonie, que nous allons décrire dans cette partie. Nous avons pu construire ce modèle simplifié des phéromones modificatrices avec des collègues  biologistes de l'INRAE, à Avignon \textit{Comment citer Yves et Cedric ici ?}. Cette Figure présente les trois principales phéromones modificatrices connues à ce jour. À gauche, les phéromones E-$\beta$-Ocimene (que nous apellerons désormais Ocimene pour plus de clarté) sont majoritairement émisent de très jeunes larves (moins de 3 jours). Ces phéromones réduisent le développement ovarien des ouvrières et déclenche une forte hausse de butinage de pollen de la part des butineuses, qui se concentrent habituellement sur le nectar \cite{maisonnasse_e-b-ocimene_2010}. Ensuite, le 9-ODA, un des composants des puissantes phéromones de reine, ralenti lui aussi le développement ovarien des butineuses mais réduit aussi les concentrations en Hormone Juvénile (HJ) chez les ouvrières. Une hypothèse répandue est de considérer que l'HJ sécrétée par la Corpora Allata permet d'altérer le fonctionnement des GH, dictant ainsi leur utilité pour les nourrices ou les butineuses. Typiquement, la transition de nourrice à butineuse se fait en une vingtaine de jours. La colonie suit ce qu'on appelle le Polyéthisme d'Âge : les adultes ayant le même âge réalisent les mêmes activités. Or, il a été montré que ce polyéthisme est souple, et que dans les bonnes conditions, une abeille peut aller butiner dès ses 5 jours, au lieu de la vingtaine habituelle. Ce mécanisme est lié à une phéromone, émise par toute la colonie, l'Ethyle Oléate (EO). La réduction d'HJ provoquée par l'Ocimene et le 9-ODA permet donc de maintenir ces ouvrières à une physiologie de nourrices.
			
			 L'Ethyle Oleate est retrouvée majoritairement sur le couvain et la reine, elle est aussi retrouvée chez les butineuses. Lorsque cette phéromone est injectée en grande quantité à des abeilles adultes, il a été montré que celles-ci arrêtent le butinage et voient leur taux d'HJ diminuer. L'Ethyle Oléate n'est pas une phéromone volatile, elle est majoritairement transmise par contact, principalement lors d'échange de nourriture et de nettoyage mutuel : lorsqu'une abeille en nettoie une autre, ou lorsqu'une nourrice nettoie une larve. Elle serait aussi transmissible sur de courtes distances par évaporation. On observe sur la Figure \ref{phMod} que l'EO favorise le développement des GH, qui elles même réduisent et son altérées par l'HJ. Nous y observons aussi que l'EO ralenti le développement ovarien des ouvrières. Dans le cas classique du polyéthisme d'âge, les jeunes abeilles effectuent un travail de nourrices, et les plus âgées butinent. Mais, comme nous venons de le voir, cet âge est souple : Une nourrice peut accélérer sont vieillissement, et une butineuse peut même l'inverser, afin de s'adapter aux besoins changeant de la colonie. C'est pour ceci que nous parlerons ici d'âge physiologique, opposé à l'âge réel. Une abeille avec un faible âge physiologique possède les GH nécessaires aux nourrices, et les plus âgées physiologiques possèdent les GH et les muscles nécessaires au vol et au butinage. 
			
		\paragraph{}
		Mécanismes de l'auto organisation
			Rôles et fonctions physiologiques associées.\\
			Différents rôles de différentes phéromones.\\
			Modélisation estimée hypothèse : \\
			Cas classique de la vie d'une abeille.\\
			Quels changements provoquent quelles réactions dans le métabolisme de l'abeille, et donc dans la répartition du travail.
			
	