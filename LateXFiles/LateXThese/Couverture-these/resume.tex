\markboth{}{}
% Plus petite marge du bas pour la quatrième de couverture
% Shorter bottom margin for the back cover
\newgeometry{inner=30mm,outer=20mm,top=40mm,bottom=20mm}

%insertion de l'image de fond du dos (resume)
%background image for resume (back)
\backcoverheader

% Switch font style to back cover style
\selectfontbackcover{ % Font style change is limited to this page using braces, just in case

\titleFR{Visualisation et Interactions avec une Colonie d'Abeilles Virtuelle : Simulation, Complexité et Pédagogie.}

\keywordsFR{Systèmes complexes, Systèmes multi-agents, Auto-organisation, Colonie d'abeilles, Visualisations et interactions immersives}

\abstractFR{Nous décrivons dans cette thèse une simulation de colonie d'abeilles. Nous proposons un modèle de répartition des tâches à base de seuils agrémenté de concepts de motivations internes permettant d'élargir le champ d'actions de ces modèles. Un agent réactif peut ainsi décider d'interrompre une action en cours en fonction de ses performances. Notre première version de colonie d'abeilles virtuelle demande à nos agents de se répartir automatiquement entre deux activités principales, le soin au couvain et le butinage, adaptant leurs physiologies à l'aide d'hormones et de phéromones. 
Différents moyens de visualisations et d'interactions avec cette simulation sont proposés par une application interactive découplée du simulateur, échangeant tous deux des informations via le réseau. Un graphique en 3 dimensions permet de rendre compte de l'état physiologique de chacun des agents, rendant visibles les mécanismes complexes régissant l'auto-organisation de notre colonie virtuelle. Une expérimentation réalisée en coopération avec le \textit{GDSA29} nous encourage à poursuivre nos efforts tant au niveau simulation qu'au niveau visualisations et interactions.}



\titleEN{Visualisation and Interactions with a Virtual Honey Bee Colony : Simulation, Complexity and Pedagogy.}

\keywordsEN{Complex Systems, Agent based simulation, Self-organisation, Honey bee colony, Immersive Visualisations and Interactions}

\abstractEN{In this thesis we describe a simulation of a honey bee colony. We propose a threshold based task allocation model, enhanced by internal motivation concepts, which allow to widen the scope of actions of these models. A reactive agent can thus decide to interrupt an action in progress depending on its performance. Our first version of a virtual honey bee colony asks our agents to automatically divide themselves between two main activities, brood care and foraging, adapting their physiology with the help of hormones and pheromones. 
Different means of visualisation and interaction with this simulation are proposed by an interactive application decoupled from the simulator, both exchanging information via the network. A 3-dimensional graph shows the physiological state of each agent, making visible the complex mechanisms governing the self-organisation of our virtual colony. An experiment carried out in cooperation with the \textit{GDSA29} encourages us to continue our efforts in both the simulation, visualisation and interaction domains.}

}

% Rétablit les marges d'origines
% Restore original margin settings
\restoregeometry
