\markboth{}{}
% Plus petite marge du bas pour la quatrième de couverture
% Shorter bottom margin for the back cover
\newgeometry{inner=30mm,outer=20mm,top=40mm,bottom=20mm}

%insertion de l'image de fond du dos (resume)
%background image for resume (back)
\backcoverheader

% Switch font style to back cover style
\selectfontbackcover{ % Font style change is limited to this page using braces, just in case

\titleFR{Visualisation et Interactions avec une Colonie d'Abeilles Virtuelle : Simulation, Pédagogie et Complexité.}

\keywordsFR{Système multi-agents, Auto-organisation, Colonie d'abeilles, Visualisation et interactions, Environnement immersif}

\abstractFR{Nous décrivons dans cette thèse nos travaux sur une simulation de colonie d'abeilles. Nous y proposons un modèle de répartition des tâches à base de seuils agrémenté de concepts de motivations internes permettant d'élargir le champs d'actions de ces modèles. Un agent réactif peut ainsi décider d'interrompre une action en cours en fonction de ses performances dans celle-ci. Nous décrions ensuite notre implémentation d'une première itération de colonie d'abeilles virtuelles fortement basés sur les informations présentes en biologie, demandant à nos agents de se répartir dynamiquement et automatiquement entre deux activités principales, le soin au couvain et le butinage. 
Nous proposons ensuite différents moyen de visualisation et d'interaction avec cette simulation en utilisant une application interactive découplé du simulateur, échangeant tout deux des informations dans les deux sens via le réseau. Nous proposons un graphique en 3 dimensions permettant de rendre compte de l'état physiologique de chacun des agents, afin de servir d'outil pour la vulgarisation des mécanismes complexes régissant l'auto-organisation de notre colonie virtuelle. Une expérimentation réalisée en coopération avec le \textit{GDSA29} nous encourage à continuer dans cette direction, afin de proposer une simulation toujours plus complète et précise d'une colonie, tout en améliorant ses différents moyens d'interactions et de visualisation.}



\titleEN{Visualisation and Interactions with a Virtual Honey Bee Colony : Simulation, Pedagogy and Complexity.}

\keywordsEN{de 3 \`{a} 6 mots clefs}

\abstractEN{Eius populus ab incunabulis primis ad usque pueritiae tempus extremum, quod annis circumcluditur fere trecentis, circummurana pertulit bella, deinde aetatem ingressus adultam post multiplices bellorum aerumnas Alpes transcendit et fretum, int, sed adversando iurgandoque cum parum congrueret, eum ad rabiem potius evibrabat, Augustum actus eius exaggerando creberrime
docens, idque, incertum qua mente, ne lateret adfectans. quibus mox Caesar acrius efferatus, velut contumaciae quoddam vexillum altius erigens, sine respectu salutis alienae vel suae ad vertenda opposita instar rapidi fluminis irrevocabili impetu ferebatur.
Hae duae provinciae bello quondam piratico catervis mixtae praedonum.}

}

% Rétablit les marges d'origines
% Restore original margin settings
\restoregeometry
