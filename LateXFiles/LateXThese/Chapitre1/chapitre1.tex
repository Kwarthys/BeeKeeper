\chapter{État de l'Art: Insectes Sociaux et Simulation Multi-Agents}
	\section{Complexité et Colonie d'abeilles, Biologie et Système complexe}
			Synthèse des connaissances, notamment de notre visite à Avignon et ses 43 degrés à l'ombre.\\
		\subsection{Complexité}
		\subsection{Auto-Organisation de la Colonie}
			Afin d'assurer le bon fonctionnement et l'épanouissement de la colonie, chaque individu la composant doit effectuer un certain nombre de tâches lorsqu'elle sont nécessaires. Contrairement à nous, ces individus ne disposent d'aucun contrôle central. Personne pour surveiller la température et prévenir une équipe que pendant un certain temps elle serait en charge de la réguler. Pas de chef pour donner des ordres.
			
			Ainsi, c'est chaque individu qui doit en quelque sorte, tout surveiller, et être prêt a adapter son activité en fonction de ses perceptions. Pour reprendre l'exemple de la température, lorsqu'un individu décide qu'il fait un peu trop chaud, il va alors commencer à refroidir ses alentours. Dans le cas des abeilles ce rafraichissement peut être réalisé soit en allant battre des ailes à l'entrée de la ruche afin de créer un courant d'air, soit en vaporisant de l'eau à l'intérieur, pour directement abaisser la température. 		
			
			
			\subsubsection{Différentes Régulations}
			\begin{itemize}
				\item Thermo-Régulation basée sur la perception locale de température et sur les différences entre individus.
				\item Sélection des meilleures sources de nourritures par les butineuses lors du recrutement.
				\item Régulation de l'âge du premier butinage en fonction des demandes du couvain.
			\end{itemize}

			\paragraph{}			
			Un des points clés de cette organisation est que chaque individu a des "tolérances" légèrement différentes. Une diversité qui provient notamment du fait que même si les abeilles ont en général toutes la même mère, elles n'ont pas toute le même père, on parle de "demie sœurs". En effet, lors de son unique vol nuptial, la future reine s'accouple avec des dizaines de mâles (qui meurent et tombent au sol juste après). Cette grande diversité génétique et les seuils de tolérances différents apportent une plus grande stabilité, tant dans les activités des abeilles que dans les paramètres contrôlés. Il a en effet été montré que l'exceptionnelle capacité des abeilles à maintenir la température à 36°C au niveau du couvain\footnote{très sensible, une variation d'un degré peut condamner le couvain et mettre la colonie en péril} est en partie due au fait que certaines abeilles vont commencer à refroidir la ruche assez tôt, alors que d'autres ne le feront que lorsque la température est déjà trop élevée. A l'extrême, certaine abeilles refroidissent la ruche en même temps que d'autres plus frileuses la réchauffent. Cet affrontement qui peut paraitre contre productif permet en réalité à la température d'être extrêmement stable, malgré les variations parfois conséquente de la température extérieure.
			
			\subsubsection{Sélection des meilleures sources de nectar}
			Lors de la collecte du nectar, les butineuses sont très sélectives et préfèrent de très loin les hautes teneurs en sucres. La colonie utilise donc un système de recrutement, presque de recommandation, afin d'allouer ses effectifs de butineuses de manière optimale et dynamique, préférant les sources proches et sucrées aux sources éloignées et peu sucrées.
			
			Ce mécanisme de recrutement se compose de trois étapes. Certaines butineuses plus téméraires vont spontanément quitter la ruche sans destination, dans le seul but de trouver une nouvelle source de nectar. Ces butineuses sont appelées des "scouts", ou des "éclaireurs". Une fois sur une source de nectar, il est temps d'en juger la qualité, c'est la deuxième étape. L'abeille va récolter le nectar, et estimer sa teneur en sucre. Si cette teneur lui convient, elle va rentrer à la colonie et commencer la troisième étape, le point clé, le recrutement.
			
			Une fois rentrée, l'abeille recruteuse va se mettre à danser la désormais célèbre "Waggle Dance", en forme de 8. Cette danse à un objectif double. Le premier : communiquer la position de la source par rapport au soleil, pour permettre aux autres de la retrouver. Le second est plus indirect : Plus la source est sucrée, plus l'abeille va danser longtemps, et même répéter la danse dans plusieurs endroits de la ruche. Une danse plus longue offre plus de temps à d'autres abeilles de venir la suivre et apprendre la position de cette nouvelle source. Ainsi, plus la source est sucrée, plus la recruteuse va communiquer la position à de nombreuses butineuses.
			
			Une fois recrutée, les nouvelles butineuses vont se rendre a la source et répéter le processus : Collecter, juger, rentrer et recruter. La encore, la diversité est clé, des abeilles moins portées sur la communication vont passer moins de temps à recruter, afin de maximiser le temps de butinage, il y a tout de même des dizaines de milliers de bouches à nourrir ! De plus, certaines butineuses sont moins difficiles que d'autres, elles vont donc communiquer des sources de faible qualité et y maintenir un faible contingent. Ce contingent alors moins utile dans l'immédiat sert de surveillance, car les teneurs en sucres des différents nectars peuvent fortement varier selon les saisons mais aussi pendant les périodes de la journée. Une source de faible qualité peut alors devenir une source extrêmement intéressante en quelques heures. Le groupe alors déjà présent peut observer ce changement et déjà danser dans toute la colonie pour avertir les autres, gagnant ainsi un temps précieux de re-découverte de la source mais aussi le temps d'amorcer une réponse conséquente. Gagner du temps sur une réaction exponentielle est toujours extrêmement précieux.
			
		\subsection{Pheromones et Physiologie}
			Afin d'obtenir cette auto-organisation, la colonie s'appuie sur différents mécanismes incluant de nombreuses boucles de rétro-actions. Des perceptions directes des différents individus permettent d'effectuer une partie de cette organisation, comme la température ou la concentration en sucre de leur nourriture. Mais la colonie s'appuie aussi sur des mécanismes indirects, physiologiques, qui prennent place grâce à différentes hormones et phéromones. Nous étudions ici en détail l'importance physiologique des glandes hypopharyngiennes (GH) et de la Corpora Allata. Voici un modèle simplifié des connaissances biologiques que nous avons à l'heure actuelle.
			
			Les GH permettent aux abeilles s'occupant du couvain de transformer le pollen et le nectar en une substance riche destinée aux larves. Elles permettent aussi aux butineuses de traiter chimiquement le nectar, le rendant utilisable pour les nourrices, transformable en miel et même consommable directement par les autres adultes. Or, ces deux comportements sont incompatibles, les GH subissent une modification physiologique pour effectuer l'une ou l'autre de ces fonctions.
			
			\paragraph{}
			La Corpora Allata permet aux adultes de sécréter une hormone appelée Hormone Juvénile(HJ). Cette hormone est retrouvée en grande quantité chez les butineuses, et en faible quantité chez les nourrices. Une hypothèse répandue est de considérer que la HJ sécrétée par la Corpora Allata permet d'altérer le fonctionnement des GH, dictant ainsi leur utilité pour les nourrices ou les butineuses. Typiquement, la transition de nourrice à butineuse se fait en une vingtaine de jours : La colonie suit ce qu'on appelle le Polyéthisme d'Âge, les adultes ayant le même âge réalisent les mêmes activités. Or, il a été montré que ce polyéthisme est souple, et que dans les bonnes conditions, une abeille peut aller butiner dès ses 5 jours, au lieu de la vingtaine habituelle.
			
			Ce mécanisme a été lié à une phéromone, émise par toute la colonie, l'Ethyle Oléate (EO). Retrouvée majoritairement sur le couvain et la reine, elle est aussi retrouvée chez les butineuses. Lorsque cette phéromone est injectée en grande quantité à des abeilles adultes, il a été montré que celles-ci arrêtent le butinage et voient leur taux d'HJ diminuer.
			
			L'Ethyle Oléate n'est pas une phéromone volatile, elle est majoritairement transmise par contact, principalement lors d'échange de nourriture et de nettoyage mutuel : lorsqu'une abeille en nettoie une autre, ou lorsqu'une nourrice nettoie une larve. Elle serait aussi transmissible sur de courtes distances par évaporation, mais nous avons décidé d'ignorer ce mécanisme pour l'instant.
			
			
			\paragraph{}
			Dans le cas classique du polyéthisme d'âge, les jeunes abeilles sont nourrices, et les plus âgées sont butineuses. Mais, comme nous venons de le voir, cet âge est souple : Une nourrice peut accélérer sont vieillissement, et une butineuse peut même l'inverser. C'est pour ceci que nous parlerons ici d'âge physiologique, opposé à l'âge réel. Une abeille avec un faible âge physiologique possède les GH nécessaires aux nourrices, et les âgées physiologiques possèdent les GH et les muscles nécessaires au vol et au butinage. 
			
		\paragraph{}
		Mécanismes de l'auto organisation
			Rôles et fonctions physiologiques associées.\\
			Différents rôles de différentes phéromones.\\
			Modélisation estimée hypothèse : \\
			Cas classique de la vie d'une abeille.\\
			Quels changements provoquent quelles réactions dans le métabolisme de l'abeille, et donc dans la répartition du travail.
			
			
			
	\section{Modèles existants de répartition des tâches}
	La division du travail se produit lorsque les agents doivent décider quelle tâche exécuter dans un environnement partagé. Les sociétés d'individus (ou d'agents) doivent trouver des moyens de répartir efficacement leur main-d'œuvre entre les tâches nécessaires pour survivre et s'étendre. En informatique, le contrôle décentralisé inspiré par les insectes sociaux a été étudié pendant des années et s'est avéré efficace dans de nombreuses applications. Dans cette section, nous allons passer en revue ce qui a été fait dans le domaine des modèles de répartition des tâches.
	
	
		\subsection{Foraging For Work  \cite{franks_foraging_1994}}
        Dans ce modèle, les différentes tâches que les agents doivent accomplir sont spatialement dispersées en zones. Les agents, en recherche active de travail, tentent d'exécuter la tâche associée à leur zone ou se déplacent de manière aléatoire. Ainsi, les zones surpeuplées "poussent" les agents vers les zones voisines offrant du travail, ce qui entraîne une division du travail. Lorsque de nouveaux agents apparaissent dans une zone spécifique et que les agents les plus âgés meurent à un certain âge, ce modèle assez simple recréé le polyéthisme d'âges : les agents du même âge effectuent globalement les mêmes tâches. Les agents nouveau-nés prennent les offres de travail dans leur zone, repoussant ainsi les agents plus âgés. En suivant des règles simples, un agent peut se rediriger dans d'autres zones s'il ne trouve pas de travail à faire dans sa zone actuelle. Pour que ce modèle fonctionne, les agents doivent avoir la capacité d'évaluer la nécessité de réaliser chaque tâche. Ce pré-requis,  ajoutée à l'hypothèse que les tâches sont dispersées dans des zones, font que ce modèle n'est pas adapté à la problématique, et doit être affiné. Il s'agit toutefois d'un bon point de départ : il peut en effet reproduire une partie du polyéthisme d'âge observé dans la colonie d'abeilles. En effet, une des hypothèses sur la migration des nourrices vers le rôle de butineuse est qu'elle est provoquée par l'émergence de nouvelle nourrices au centre de zones de couvain, repoussant ainsi les nourrices plus âgées.
        
        
		\subsection{Modèles à Seuils}
		\label{subsectionRTM}
		\subsubsection{FTM: "Fixed Threshold Model" \cite{bonabeau_natural_1999}}
        Chaque tâche a un score, représentant sa priorité. Un agent s'engage puis exécute la tâche ayant la priorité la plus élevée. Le FTM est basé sur quelques hypothèses fondatrices, dont voici la première : chaque tâche doit être associée à un stimulus. Le score de chaque tâche est calculé à partir de l'intensité du stimulus associé perçu par l'agent, généralement à l'aide d'une fonction sigmoïde. Soit $T$ la tâche évaluée par l'agent, $F(T)$ le score de la tâche $T$ et $x_T$ le stimulus associé (simple ou complexe) perçu par l'agent. Ces fonctions sigmoïdes prennent alors la forme :
			
\begin{equation}
\label{equationSigmoid}
	F(T) = \frac{x_T^n}{x_T^n + \Theta_T^n}
\end{equation}	
 Avec $n$ un entier pour la non-linéarité de la fonction (généralement $n=2$ \cite{schmickl_taskselsim_2008}) et $\Theta_T$ le seuil de la tâche, aussi appelé biais, de la sigmoïde. Vous trouverez Figure \ref{sigmoids} différentes sigmoïdes mettant en valeur l'impact du paramètre $\Theta_T$. Le seuil sert en quelque sorte de point d'ancrage : lorsque le seuil est strictement équivalent au stimulus d'entrée, alors la valeur du résultat est exactement 0,5. Ce biais est utilisé pour modifier la perception des agents : avec un biais très faible, les agents sont très sensibles au stimulus associé et s'engagent dans la tâche plus tôt que les agents avec un biais plus élevé \cite{dornhaus_task_nodate}. 
        
        Largement utilisés pour modéliser et piloter des simulations d'insectes sociaux, les modèles à seuils reposent fortement sur l'association entre tâches et stimulus. Voici les autres hypothèses fondatrices : ils supposent également que l'exécution d'une tâche diminue le stimulus qui lui est associé, et que ne pas exécuter une tâche augmente son stimulus associé. Dans le cas contraire, les agents exécuteraient constamment cette tâche, ou du moins même jusqu'à ce qu'elle ne soit plus prioritaire. Le stimulus doit être une représentation de la priorité de la tâche qui lui est associée. 
		
		\begin{figure}
		\centering
		\includegraphics[width=0.5\textwidth]{./Pictures/Figures/sigmoids.png}
		\caption{L'influence du paramètre téta ($\Theta$) sur la forme des sigmoides.}
		\label{sigmoids}
		\end{figure}
		
		Chaque tâche a également une probabilité d'interruption aléatoire évaluée à chaque pas de temps. Un agent peut avoir, par exemple, $0.5\%$ de chances d'interrompre sa tâche en cours \cite{schmickl_analysing_2008}. Lorsque c'est le cas, l'agent recherche une nouvelle tâche en utilisant les scores de chaque tâche et choisit celle au score le plus élevé. Cette interruption totalement aléatoire se repose sur les hypothèses fondatrice des modèles à seuils.
        
        \subsubsection{RTM : "Response Threshold Model", Renforcement du biais}
        Sur la base du FTM et de l'équation \ref{equationSigmoid}, différents travaux des années 90 \cite{theraulaz_response_1998,carbonell_multi-agent_1994} ont proposé de mettre en place des mécanismes de renforcement de la valeur $\Theta$, en modifiant la sensibilité des agents pendant l'exécution, formant ainsi efficacement des spécialistes. Cette mise à niveau du FTM est plus généralement appelée "Response Threshold Model" (RTM).
        
			
	\section*{Conclusion}
	Le modèle "Foraging For Work" nous intéresse car il permet de simuler un cas intéressant de la colonie, la migration des nourrices vers le butinage. En revanche, il ne nous permettra pas de simuler l'ensemble de la colonie. C'est pour ceci que nous nous tournons vers les modèles à seuils, les "RTM". Ils présentent trois hypothèses fondatrices pour fonctionner : chaque tâche doit être associée à un stimulus déclencheur, l'exécution de la tâche viens réduire l'intensité de son stimulus associé, qui augmente lorsque la tâche n'est pas (ou pas assez) exécutée par les agents.
	D'après nos connaissances sur l'abeille, nous trouvons des tâches qu'elles réalisent qui ne respectent pas ces deux conditions : pas de stimulus direct pour pousser au butinage (sauf quelques cas précis) ou à l'alimentation du couvain. Nous nous intéressons donc ici aux situations dans lesquelles ces hypothèses ne sont pas vraies. Nous décrivons dans le chapitre suivant notre modèle fondé sur un RTM et agrémenté d'un mécanisme supplémentaire basé sur la motivation interne pour gérer ces tâches ne respectant pas les hypothèses. Nous allons aussi devoir nous passer de l'interruption aléatoire proposée par ces modèles, qui ne fait sens que lorsque les deux hypothèses sont valides.
