\chapter{État de l'Art: Colonies d'Abeilles et Simulation Multi-Agents}

	Afin de comprendre et reproduire les capacités complexes des insectes sociaux, ils sont étudiés depuis une cinquantaine d'années dans le domaine des Simulations Multi-Agents (SMA). Plusieurs travaux ont ainsi pu reproduire et approcher leurs phénoménales capacités d'auto-organisation dynamique par la simulation, notamment Swarm Intelligence \cite{bonabeau_natural_1999} et bien d'autres \cite{drogoul_simulation_1993, schmickl_taskselsim_2008, dornhaus_task_1998}. Ce chapitre présente quelques modèles théoriques présents dans la littérature servant à modéliser ces systèmes, ainsi que quelques applications de ceux-ci, dans le cadre de la simulation d'insectes sociaux, et d'autres.
			
			
	\section{Modèles existants de répartition des tâches}
	La division du travail se produit lorsque les agents doivent décider quelle tâche exécuter dans un environnement partagé. Les sociétés d'individus (ou d'agents) doivent trouver des moyens de répartir efficacement leur main-d'œuvre entre les tâches nécessaires pour survivre et s'étendre. En informatique, le contrôle décentralisé inspiré par les insectes sociaux a été étudié pendant des années et s'est avéré efficace dans de nombreuses applications. Dans cette section, nous allons passer en revue ce qui a été fait dans le domaine des modèles de répartition des tâches.
	
	
		\subsection{Foraging For Work  \cite{franks_foraging_1994}}
        Dans ce modèle, les différentes tâches que les agents doivent accomplir sont spatialement dispersées en zones. Les agents, en recherche active de travail, tentent d'exécuter la tâche associée à leur zone ou se déplacent de manière aléatoire. Ainsi, les zones surpeuplées "poussent" les agents vers les zones voisines offrant du travail, ce qui entraîne une division du travail. Lorsque de nouveaux agents apparaissent dans une zone spécifique et que les agents les plus âgés meurent à un certain âge, ce modèle assez simple recrée le polyéthisme d'âges : les agents du même âge effectuent globalement les mêmes tâches. En effet, la zone voyant des agents naitre va contenir plus d'agents qu'elle n'offre de travail, là ou du même temps une zone voyant des agents mourir sera dans le cas inverse. Ainsi, nous obtenons une zone de naissance qui aura tendance à repousser les agents, et la zone de "mort" va les attirer, car elle a besoin de main d'œuvre. Nous obtenons une répartition spatiale liée à l'âge des individus, nous retrouvons donc bien une forme de polyéthisme car les zones sont associées a des tâches. Ce modèle nous intéresse particulièrement pour cette capacité, en effet une des hypothèses sur la migration des nourrices vers le rôle de butineuse est qu'elle est provoquée par l'émergence de nouvelles nourrices au centre de zones de couvain, repoussant ainsi les nourrices plus âgées vers d'autres activités, ce que le \textit{Foraging For Work} est tout a fait à même de recréer.
        
        Ce modèle repose sur deux hypothèses fortes :
        
        1. Les agents doivent avoir la capacité d'évaluer les besoins de chaque tâche, leur priorité en quelque sorte. 
        
        2. Les tâches sont dispersées en zones géographiques définies.
        
        
		\subsection{Modèles à Seuils}
		\label{subsectionRTM}
		\subsubsection{FTM : "Fixed Threshold Model" \cite{bonabeau_natural_1999}}
        Avec ce modèle, chaque tâche a un score, représentant sa priorité. Un agent s'engage puis exécute la tâche ayant la priorité la plus élevée. Le FTM est basé sur quelques hypothèses fondatrices, dont voici la première : chaque tâche est associée à un stimulus. Le score de chaque tâche est calculé à partir de l'intensité du stimulus associé perçu par l'agent, généralement à l'aide d'une fonction sigmoïde. Soit $T$ la tâche évaluée par l'agent, $F(T)$ le score de la tâche $T$ et $x_T$ le stimulus associé (simple ou complexe) perçu par l'agent, ces fonctions prennent alors la forme :
			
\begin{equation}
\label{equationSigmoid}
	F(T) = \frac{x_T^n}{x_T^n + \Theta_T^n}
\end{equation}	
 avec $n$ un entier pour la non-linéarité de la fonction (généralement $n=2$ \cite{schmickl_taskselsim_2008}) et $\Theta_T$ le seuil de la tâche, aussi appelé biais, de la sigmoïde. La Figure \ref{sigmoids} présente différentes sigmoïdes mettant en valeur l'impact du paramètre $\Theta_T$. Le seuil sert en quelque sorte de point d'ancrage : lorsque le seuil est strictement équivalent au stimulus d'entrée, alors la valeur du résultat est exactement 0,5. Ce biais est utilisé pour modifier la perception des agents : avec un biais très faible, les agents sont très sensibles au stimulus associé et s'engagent dans la tâche plus tôt que les agents avec un biais plus élevé \cite{dornhaus_task_nodate}. 
        
        Largement utilisés pour modéliser et piloter des simulations d'insectes sociaux, les modèles à seuils reposent fortement sur l'association entre tâches et stimulus. Voici les autres hypothèses fondatrices : ils supposent également que l'exécution d'une tâche diminue le stimulus qui lui est associé, et que ne pas exécuter une tâche augmente son stimulus associé. Dans le cas contraire, les agents exécuteraient constamment cette tâche, ou du moins même jusqu'à ce qu'elle ne soit plus prioritaire. Le stimulus doit être une représentation de la priorité de la tâche qui lui est associée. 
        
        Par exemple, Bonabeau et al. \cite{bonabeau_quantitative_1996} utilisèrent un FTM pour modéliser la répartition du travail au sein d'une espèce de fourmi contenant deux types d'individu aux caractéristiques physiques très différentes \cite{wilson_relation_1984}. Appelés "Majors" et "Minors" dans leurs travaux, ces castes correspondent respectivement a ce que nous pourrions voir comme de grands soldats et de petites ouvrières. Dans la nature, il a été observé que les ouvrières travaillent en permanence, alors que les soldats travaillent seulement lorsque la demande est trop forte par rapport au nombre d'ouvrières présentes. Ces deux castes ont alors été modélisées, chacune avec un seuil différent pour une même tâche abstraite. Ainsi, les ouvrières ont reçu un seuil très faible, elles s'engagent donc dans la tâche même lorsque le stimulus déclencheur est relativement faible. À l'inverse, les soldats ont un seuil élevé, elles nécessitent donc un stimulus déclencheur très intense pour engager la tâche. Ainsi, lorsque le nombre d'ouvrières est suffisant pour maintenir le stimulus à un niveau faible (elles sont suffisamment nombreuse par rapport à la demande), les soldats ne travaillent pas car le stimulus n'atteindra jamais une valeur suffisamment élevée. En revanche, lorsque des ouvrières sont retirées de la simulation (ou de la colonie), le peu qui reste ne parvient plus a maintenir le stimulus bas : la demande dépasse l'offre. Le stimulus grimpe donc régulièrement, jusqu'à atteindre la seuil déclencheur pour les soldats, qui se mettent alors au travail. Ce qui est intéressant c'est que lors de la réintroduction des ouvrières, les soldats arrêterons rapidement de travailler, le stimulus déclencheur redevenant trop faible. On obtient donc un magnifique exemple d'auto-organisation sans aucun contrôle central, sur une seule tâche et avec deux populations aux seuils fixes, mais différents.
		
		\begin{figure}
		\centering
		\includegraphics[width=0.5\textwidth]{./Pictures/Figures/sigmoids.png}
		\caption{L'influence du paramètre thêta ($\Theta$) sur la forme des sigmoides.}
		\label{sigmoids}
		\end{figure}
		
		Chaque tâche a également une probabilité d'interruption aléatoire évaluée à chaque pas de temps. Par exemple, un agent peut avoir $0.5\%$ de chances d'interrompre sa tâche en cours, à chaque pas de temps \cite{schmickl_analysing_2008}. Lorsque c'est le cas, l'agent recherche une nouvelle tâche en utilisant les scores de chaque tâche et choisit celle au score le plus élevé. Cette interruption totalement aléatoire se repose sur les hypothèses fondatrice des modèles à seuils.
        
        \subsubsection{RTM : "Response Threshold Model", Renforcement du biais}
        Sur la base du FTM et de l'équation \ref{equationSigmoid}, différents travaux des années 90 \cite{theraulaz_response_1998,carbonell_multi-agent_1994} ont proposé de mettre en place des mécanismes de renforcement de la valeur $\Theta$, en modifiant la sensibilité des agents pendant l'exécution, formant ainsi efficacement des spécialistes. Cette mise à niveau du FTM est plus généralement appelée "Response Threshold Model" (RTM).
        
        Par exemple, Cicirello et Smith \cite{cicirello_wasp-like_2004} utilisèrent un modèle à seuils pour résoudre un problème d'allocation de ressource : une ligne d'assemblage de \textit{General Motors} qui doit peindre des camions tout juste assemblés, de différentes couleurs. Chaque compartiment de peinture est alors vu comme un agent qui possède une tâche par couleur de camion. Ainsi, une tâche consiste à peindre un camion d'une couleur, et changer de couleur signifie changer de tâche. Chaque tâche possède un seuil variable, permettant d'exprimer à la fois la spécialisation d'un compartiment pour une couleur, mais aussi indirectement d'exprimer le coût du changement de couleur. En effet, lors d'un changement de couleur, beaucoup de temps est perdu car il faut purger tout le système du compartiment, gâchant du même coup une bonne quantité de peinture. L'idée est donc de minimiser les coûts en peinture ainsi que le temps pour peindre une grande série de camions de couleurs différentes et inconnues \textit{a priori}. Une file de camions à peindre arrive en entrée et les compartiments doivent en accepter certains pour les peindre. Un compartiment ajuste les seuils de ses tâches à chaque pas de temps. Ainsi, il diminue la tâche correspondant à sa couleur actuelle, augmentant ses chances d'accepter de peindre un camion de cette couleur, et à l'inverse augmente les seuils de toutes ses autres tâches. Lorsqu'un compartiment n'a aucun camion à peindre, il diminue alors tous ses seuils, de plus en plus vite avec le temps qui passe.
        
        De cette manière, Cicirello et Smith arrivent à grandement limiter le nombre de changement de couleurs nécessaires, tout en conservant un rendement proche des méthodes traditionnelles comme l'approche basée sur les lois du marché, que nous allons désormais aborder rapidement. PEUT ETRE
        
        
        \section{Modèles de Répartitions des Tâches et Insectes Sociaux}
        Nous nous intéressons ici aux applications pratiques de modèles théoriques, principalement ceux que nous venons de décrire. 
        
			
	\section*{Conclusion}
	Le modèle "Foraging For Work" nous intéresse car il permet de simuler un cas intéressant de la colonie, la migration des nourrices vers le butinage. En revanche, il ne nous permettra pas de simuler l'ensemble de la colonie. C'est pour ceci que nous nous tournons vers les modèles à seuils : les "RTM". Ils présentent trois hypothèses fondatrices pour fonctionner : chaque tâche doit être associée à un stimulus déclencheur, l'exécution de la tâche vient réduire l'intensité de son stimulus associé, qui augmente lorsque la tâche n'est pas (ou pas assez) exécutée par les agents.
	D'après nos connaissances sur les abeilles, nous trouvons des tâches qu'elles réalisent qui ne respectent pas ces deux conditions : pas de stimulus directs pour pousser au butinage (sauf quelques cas précis) ou à l'alimentation du couvain. Nous nous intéressons donc ici aux situations dans lesquelles ces hypothèses ne sont pas vraies. Aussi, si certaines tâches, notament celles liées au couvain, peut être séparée en zone définie dans la ruche, ce n'est pas le cas de la majorité : le modèle FFW ne pourra donc pas nous aider à modéliser l'ensemble de la colonie.
	
	 Nous décrivons dans le chapitre suivant notre modèle fondé sur un RTM et agrémenté d'un mécanisme supplémentaire basé sur la motivation interne pour gérer ces tâches ne respectant pas les hypothèses. Nous allons aussi devoir nous passer de l'interruption aléatoire proposée par ces modèles, qui ne fait sens que lorsque les deux hypothèses sont valides.
